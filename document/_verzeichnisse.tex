\section{Verzeichnisse}
\subsection{Inhaltsverzeichnis}
Ein Inhaltsverzeichnis wird an der gewünschten Stelle mit dem Kommando 
\lstinline$ \tableofcontents $ eingefügt. Dieses enthält alle nummerierten 
Überschriften, also diese, welche nicht mit einem $*$ versehen wurden.
\subsection{Abkürzungsverzeichnis}

\subsection{Tabellenverzeichnis}
Das Tabellenverzeichnis verweist auf die Tabelleneigenschaft 
\lstinline$ \caption{Beschreibung} $. Es wird mit 
\lstinline $ \listoftables $ eingebunden.

\subsection{Abbildungsverzeichnis}
Das Abbildungsverzeichnis erstellt man mit \lstinline$ \listoffigures $ . 
Dieses verweist auf die zu den Abbildungen referenzierten 
\lstinline $ \caption[Titel im Verzeichnis]{Anderer Titel} $.
\subsection{Listingverzeichnis}
Ähnlich verhält es sich bei dem Listing Verzeichnis, dieses referenziert auf 
das \lstinline $ [caption={meine Beschreibung}, ...] $ Attribut im Listing.
Das Verzeichnis wird mit dem Befehl \lstinline $ \lstlistoflistings $ 
eingebunden.
\subsection{Anhangsverzeichnis}
\subsection{Quellenverzeichnis}