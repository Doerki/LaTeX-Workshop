\section{Verzeichnisse}
\subsection{Inhaltsverzeichnis}
Ein Inhaltsverzeichnis wird an der gewünschten Stelle mit dem Kommando 
\lstinline$ \tableofcontents $ eingefügt. Dieses enthält alle nummerierten 
Überschriften, also diese, welche nicht mit einem $*$ versehen wurden.
\subsection{Abkürzungsverzeichnis}

\subsection{Tabellenverzeichnis}
\subsection{Abbildungsverzeichnis}
Das Abbildungsverzeichnis erstellt man mit \lstinline$ \listoffigures $ . 
Dieses verweist auf die zu den Abbildungen referenzierten 
\lstinline $ \caption[Titel für das Verzeichnis]{Anderer Titel} $
\subsection{Listingverzeichnis}
\subsection{Anhangsverzeichnis}
\subsection{Quellenverzeichnis}



 %Abbildungsverzeichnis
\listoftables %Tabellenverzeichnis
\lstlistoflistings %Listingverzeichnis