\section{Grundaufbau}
\subsection{Dokumentenstruktur}
Ein \LaTeX Dokument besteht zunächst aus zwei grundlegenden Teilen. Zum einen 
gibt es den \texttt{Header}-Teil und zum anderen den \texttt{document}-Teil.
\\
Der  \texttt{Header}-Teil beinhaltet grundlegende Layout Einstellungen, 
eingebundene Pakete und weitere Definitionen.
\\
Der Body des Dokuments enthält die eigentlichen Daten, die in ein Dokument 
eingefügt werden sollen.
\\ In der Abbildung \ref{lst:main:main} ist ein grundlegender 
Dokumentenausbau dargestellt:
\lstinputlisting[label=lst:main:main]{./src/main/main.tex}

\subsection{Grundlegende Formatierungen}
Wie auch in anderen Texteditoren lassen sich in \LaTeX einzelne Wörter oder 
Codeblöcke verschieden formatieren: \\
\newline
\begin{flushleft}
\textbf{Ich bin fett geschrieben} \hspace{1cm}
\textit{Ich bin kursiv geschrieben} \hspace{1cm}
\underline{Ich bin unterstrichen} \\
\texttt{Das ist Schreibmaschinenschrift} \hspace{1cm}
\underline{\textbf{Ich bin Fett und unterstrichen}} 
\flushleft Ich werde linksbündig dargestellt \\
\flushright Ich befinde mich rechts \\
\center Ich bin in der Mitte
\end{flushleft}



\lstinputlisting[label=lst:main:format1]{./src/main/format1.tex}

Diese und viele weitere Formatierungen können ebenso als Blockformatierung 
verwendet werden:

\lstinputlisting[label=lst:main:format2]{./src/main/format2.tex}

\subsection{Umbrüche und Freiräume}
In \LaTeX \ gibt es verschiedene Möglichkeiten, Zeilen, Seiten, oder 
Abschnittsumbrüche herbeizuführen. Außerdem kann man Abstände zwischen Blöcken 
definieren:
\lstinputlisting[label=lst:main:umbruch]{./src/main/umbruch.tex}

\subsection{Überschriften / Abschnitte}
Überschriften bzw. Abschnitte lassen sich mit den folgenden Kommandos einfügen:
\lstinputlisting[label=lst:main:sections]{./src/main/sections.tex}
Die Überschriften werden automatisch dem Inhaltsverzeichnis hinzugefügt, sofern 
dieses eingebunden wurde und die Überschriften nicht mit einem $*$ 
gekennzeichnet sind.
\subsection{Farben}

Um in \LaTeX verschiedene Farben zu verwenden benötigt man zunächst
\lstinline$ \usepackage{color} $ bzw. \lstinline$ \usepackage{xcolor} $ im 
\texttt{Header}-Bereich des Dokumentes.
\\
Anschließend können Farben folgendermaßen benutzt werden:
\begin{flushleft}
\pagecolor{white}   % weißer Seitenhintergrund
\color{blue}         % blauer Text
Dieser Text ist
\textcolor{green}{\textbf{BLAU}}!    % gruener Text "BLAU"
\colorbox{orange}{Box mit orangen Hintergrund}   % roter Text in oranger Box
\fboxrule1mm
\definecolor{rahmen}{rgb}{.7,1,.7}      % hellgruener Rahmen
\definecolor{grund}{gray}{.8}           % grauer Box-Hintergrund
\definecolor{schrift}{cmyk}{.4,1,1,0}   % hellroter Text
\color{schrift}
\fcolorbox{rahmen}{grund}{Box mit hellgruenem Rahmen und grauem Hintergrund; 
Der Text ist hellrot}\\
\end{flushleft}
\lstinputlisting[label=lst:main:color]{./src/main/color.tex}


\subsection{Listen}
In LaTeX ist es möglich nummerierte und nicht nummerierte Listen darzustellen:

\begin{tabular}{c|c}
nummerierte Liste & nicht nummerierte Liste \\ 
\hline  
\begin{itemize}
\item
\end{itemize}	
	&  \\ 
	
\hline 
	&  \\ 
	
\end{tabular} 