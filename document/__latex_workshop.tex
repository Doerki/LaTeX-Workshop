\documentclass[
    numbers=noenddot, %Keine Punkte am Ende des TOC
    toc=flat, %Flache TOC
    12pt, % Schriftgröße
    titlepage, % es wird eine Titelseite verwendet
    %parskip=half, % Abstand zwischen Absätzen (halbe Zeile)
    listof=totoc, % Verzeichnisse im Inhaltsverzeichnis aufführen
    %^enlargefirstpage,
    bibliography=totoc, % Literaturverzeichnis im Inhaltsverzeichnis aufführen
    %index=totoc, % Index im Inhaltsverzeichnis aufführen
    %captions=tableheading, % Beschriftung von Tabellen unterhalb ausgeben
    %draft % Status des Dokuments (final/draft) draft hinzufügen zum anziegen 
    %%der zeilen ende
]{scrartcl}

\usepackage[utf8]{inputenc} %UTF8 nutzen
\usepackage[ngerman]{babel} %deutsch umlaute

\usepackage[a4paper,left=2.5cm,right=2.5cm, top=2cm, bottom=2.5cm]{geometry}
%Zeilenabstand anpassen
\usepackage{setspace}
%\onehalfspacing
%Neues Seiten format, bitte
\setlength{\footskip}{1.2cm}

\usepackage{fancyhdr} %Huebsche footer und header
\usepackage[bookmarks=true]{hyperref} %fuer klickbare links und url 
%formatierungen
\hypersetup{pdfborder={0 0 0}} %--rahmen um links in pdf ausschalten
%\usepackage{showframe}

\usepackage{bibgerm}
%\usepackage[superscript]{cite} %Hochgestellte Zahlen für quellen

\usepackage{multirow}
\usepackage{rotfloat}
\usepackage{tabularx}
\usepackage{selinput}
\usepackage{listings}
\usepackage{lmodern}
\usepackage{paralist}
\usepackage{array}
\usepackage{float}
\usepackage[T1]{fontenc}
\usepackage{diagbox}
\usepackage[]{pgfgantt}
\usepackage[]{palatino}
\usepackage{csquotes}
\usepackage{paralist}
% hyphenations
%\hyphenation{}



\renewcommand{\arraystretch}{1.5}
\newcommand{\versiondate}{07.01.2015}
\newcommand{\authorname}{Danny Koppenhagen}

%Verzeichnisse
\usepackage[resetlabels]{multibib}
\newcites{que}{Quellenverzeichnis}
\newcites{lit}{Literaturverzeichnis}

\usepackage{listings} %für quellcode

\usepackage{graphicx} %grafiken einbinden
%\usepackage{warpfig} % Text um Bild

\usepackage{amssymb}
%Beispiele fuer manuelle Silbentrennung
% Silbentrennung direkt mit /- oder /"
%\hyphenation{ge-wuensch-ten}

\usepackage[printonlyused]{acronym} %für Abkürzungsverzeichnis und 
%Abkürzungsverwendung

% include pdfs directly into tex
\usepackage{pdfpages}
\usepackage[section]{placeins}
% LISTINGS
\usepackage{color} % Wird benoetigt fuer listing
\usepackage{xcolor} % Wird benoetigt fuer listing

\definecolor{dkgreen}{rgb}{0,0.6,0}
\definecolor{gray}{rgb}{0.5,0.5,0.5}
\definecolor{lightgray}{gray}{0.5}
\definecolor{mauve}{rgb}{0.58,0,0.82}
\definecolor{darkgray}{rgb}{0.4,0.4,0.4}
\definecolor{purple}{rgb}{0.65, 0.12, 0.82}
\definecolor{orange}{rgb}{1, 0.3, 0}

\usepackage{listings} %fuer listings von quellcode




\lstset
{
    language={[LaTeX]TeX},
    %alsolanguage={PGF/TikZ},
    frame=single,
    framesep=\fboxsep,
    framerule=\fboxrule,
    rulecolor=\color{black},
    xleftmargin=\dimexpr\fboxsep+\fboxrule,
    xrightmargin=\dimexpr\fboxsep+\fboxrule,
    breaklines=true,
    basicstyle=\small\tt,
    keywordstyle=\color{blue}\sf,
    identifierstyle=\color{red},
    commentstyle=\color{cyan},
    backgroundcolor=\color{green!2},
    tabsize=2,
    columns=flexible,
}


\begin{document}
\pagenumbering{gobble}
%\section{Deckblatt}
\thispagestyle{empty}
\begin{center}
%\Large{A Additional Headline}\\
\end{center}
\begin{verbatim}


\end{verbatim}
\begin{center}
\textbf{\Huge{Workshop \LaTeX}}
\end{center}
\begin{verbatim}















\end{verbatim}
\begin{flushleft}
\begin{tabular}{p{5cm} l }
\textbf{Autor:} 	& \authorname \\
\textbf{Web:}  		& \url{www.d-koppenhagen.de} \\
\textbf{Github:}  	& \url{www.github.com/Doerki} \\
\end{tabular}
\end{flushleft}
%\includegraphics[scale=0.25,clip=false]{Logo/logo.pdf}

\newpage
	
\pagestyle{fancy} %eigener Seitenstil an hier
\fancyhf{} % Kopf- und Fußzeilenfelder bereinigen
\renewcommand{\headrulewidth}{0.4pt} %obere Trennlinie sichtbar setzen
\renewcommand{\footrulewidth}{0.4pt} %untere Trennlinie sichtbar setzen
	
\lhead{{\footnotesize \textsc{Inhaltsverzeichnis}}} %Kopfzeile anpassen
\tableofcontents
\newpage % Neue Seite
	
\renewcommand{\sectionmark}[1]{\markboth{#1}{}}
\rhead{}
\chead{}
\lhead{{\footnotesize \textsc{Verzeichnisse}}} %Kopfzeile anpassen	
\lfoot{}
\cfoot{}
\rfoot{}
	
\begin{acronym}[Abkuerzungsverzeichnis]
	\acro{HfTL}{Hochschule fuer Telekommunikation Leipzig}
	\acro{WebRTC}{Web Real-Time Communication}
\end{acronym}	

%\newpage
\listoffigures %Abbildungsverzeichnis
\listoftables %Tabellenverzeichnis
\lstlistoflistings %Listingverzeichnis
	
\newpage
\pagenumbering{arabic}
\setcounter{page}{1}
\parindent0pt % Keine Absatzeinrückungen
\parskip2ex %Absatzhöhe
	
\lhead{{\footnotesize \thesection \quad \textsc{\leftmark}}}  %Kopfzeile in der 
%Kaptitel und Nummerierung links steht
\chead{}
\rhead{\thepage}
	
\lfoot{\authorname}
\cfoot{}
\rfoot{\today}
	
\newpage
%%%%%%%%%%%%%%%%%%%%%%%
\section{Grundaufbau}
\subsection{Dokumentenstruktur}
Ein \LaTeX Dokument besteht zunächst aus zwei grundlegenden Teilen. Zum einen 
gibt es den \texttt{Header}-Teil und zum anderen den \texttt{document}-Teil.
\\
Der  \texttt{Header}-Teil beinhaltet grundlegende Layout Einstellungen, 
eingebundene Pakete und weitere Definitionen.
\\
Der Body des Dokuments enthält die eigentlichen Daten, die in ein Dokument 
eingefügt werden sollen.
\\ In der Abbildung \ref{lst:main:main} ist ein grundlegender 
Dokumentenausbau dargestellt:
\lstinputlisting[label=lst:main:main]{./src/main/main.tex}

\subsection{Grundlegende Formatierungen}
Wie auch in anderen Texteditoren lassen sich in \LaTeX einzelne Wörter oder 
Codeblöcke verschieden formatieren: \\
\newline
\begin{flushleft}
\textbf{Ich bin fett geschrieben} \hspace{1cm}
\textit{Ich bin kursiv geschrieben} \hspace{1cm}
\underline{Ich bin unterstrichen} \\
\texttt{Das ist Schreibmaschinenschrift} \hspace{1cm}
\underline{\textbf{Ich bin Fett und unterstrichen}} 
\flushleft Ich werde linksbündig dargestellt \\
\flushright Ich befinde mich rechts \\
\center Ich bin in der Mitte
\end{flushleft}



\lstinputlisting[label=lst:main:format1]{./src/main/format1.tex}

Diese und viele weitere Formatierungen können ebenso als Blockformatierung 
verwendet werden:

\lstinputlisting[label=lst:main:format2]{./src/main/format2.tex}

\subsection{Umbrüche und Freiräume}
In \LaTeX \ gibt es verschiedene Möglichkeiten, Zeilen, Seiten, oder 
Abschnittsumbrüche herbeizuführen. Außerdem kann man Abstände zwischen Blöcken 
definieren:
\lstinputlisting[label=lst:main:umbruch]{./src/main/umbruch.tex}

\subsection{Überschriften / Abschnitte}
Überschriften bzw. Abschnitte lassen sich mit den folgenden Kommandos einfügen:
\lstinputlisting[label=lst:main:sections]{./src/main/sections.tex}
Die Überschriften werden automatisch dem Inhaltsverzeichnis hinzugefügt, sofern 
dieses eingebunden wurde und die Überschriften nicht mit einem $*$ 
gekennzeichnet sind.
\subsection{Farben}

Um in \LaTeX verschiedene Farben zu verwenden benötigt man zunächst
\lstinline$ \usepackage{color} $ bzw. \lstinline$ \usepackage{xcolor} $ im 
\texttt{Header}-Bereich des Dokumentes.
\\
Anschließend können Farben folgendermaßen benutzt werden:
\begin{flushleft}
\pagecolor{white}   % weisser Seitenhintergrund
\color{blue}         % blauer Text
Dieser Text ist
\textcolor{green}{\textbf{BLAU}}!    % gruener Text "BLAU"
\colorbox{orange}{Box mit orangen Hintergrund}   % roter Text in oranger Box
\fboxrule1mm
\definecolor{rahmen}{rgb}{.7,1,.7}      % hellgruener Rahmen
\definecolor{grund}{gray}{.8}           % grauer Box-Hintergrund
\definecolor{schrift}{cmyk}{.4,1,1,0}   % hellroter Text
\color{schrift}
\fcolorbox{rahmen}{grund}{Box mit hellgruenem Rahmen und grauem Hintergrund; 
Der Text ist hellrot}\\
\end{flushleft}
\lstinputlisting[label=lst:main:color]{./src/main/color.tex}


\subsection{Listen}
In LaTeX ist es möglich nummerierte und nicht nummerierte Listen darzustellen:
\subsubsection{Nummerierte Liste}
\begin{enumerate}
	\item item1
	\item item1
	\begin{enumerate}
		\item item1
		\item item2
		\begin{enumerate}
				\item item1
				\item item2
			\end{enumerate}
	\end{enumerate}
\end{enumerate}
\section{Tabellen}
\subsection{Das Grundgerüst}
\begin{table}[h]
\begin{tabular}{l|l|l|l|l}
Hallo  & Das & ist & die & Überschrift \\ \hline
Inhalt &     &     &     &             \\
Inhalt &     &     &     &             \\
Inhalt &     &     &     &            
\end{tabular}
\caption{Tabellenname}
\end{table}

\lstinputlisting[label=lst:tabellen:main]{./src/tabellen/main.tex}
\section{Abbildungen}
\section{Verzeichnisse}
\subsection{Inhaltsverzeichnis}
Ein Inhaltsverzeichnis wird an der gewünschten Stelle mit dem Kommando 
\lstinline$ \tableofcontents $ eingefügt. Dieses enthält alle nummerierten 
Überschriften, also diese, welche nicht mit einem $*$ versehen wurden.
\subsection{Abkürzungsverzeichnis}
Ein Abkürzungsverzeichnis wird in folgender Art und Weise erstellt:
\lstinputlisting[label=lst:verz:acro]{./src/verz/acro.tex}

Nach dem hinzufügen der Abkürzungen kann man diese auf verschiedene Art und 
Weise im Text aufrufen;

\subsection{Tabellenverzeichnis}
Das Tabellenverzeichnis verweist auf die Tabelleneigenschaft 
\lstinline$ \caption{Beschreibung} $. Es wird mit 
\lstinline $ \listoftables $ eingebunden.

\subsection{Abbildungsverzeichnis}
Das Abbildungsverzeichnis erstellt man mit \lstinline$ \listoffigures $ . 
Dieses verweist auf die zu den Abbildungen referenzierten 
\lstinline $ \caption[Titel im Verzeichnis]{Anderer Titel} $.
\subsection{Listingverzeichnis}
Ähnlich verhält es sich bei dem Listing Verzeichnis, dieses referenziert auf 
das \lstinline $ [caption={meine Beschreibung}, ...] $ Attribut im Listing.
Das Verzeichnis wird mit dem Befehl \lstinline $ \lstlistoflistings $ 
eingebunden.
\subsection{Anhangsverzeichnis}
\subsection{Quellenverzeichnis}
%%%%%%%%%%%%%%%%%%%%%%%
	
\newpage
\pagenumbering{gobble}
\begin{flushleft}% Flattersatz
	    \begin{appendix}
	        \fancyhf{} %alle Kopf- und Fußzeilenfelder bereinigen
	        \renewcommand{\headrulewidth}{0.0pt} %obere Trennlinie unsichtbar
	        \renewcommand{\footrulewidth}{0.0pt} %untere Trennlinie unsichtbar
			%\input{_umlDiff}
		\end{appendix}
	\end{flushleft}
\end{document}
